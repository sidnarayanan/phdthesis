% $Log: abstract.tex,v $
% Revision 1.1  93/05/14  14:56:25  starflt
% Initial revision
% 
% Revision 1.1  90/05/04  10:41:01  lwvanels
% Initial revision
% 
%
%% The text of your abstract and nothing else (other than comments) goes here.
%% It will be single-spaced and the rest of the text that is supposed to go on
%% the abstract page will be generated by the abstractpage environment.  This
%% file should be \input (not \include 'd) from cover.tex.

Astrophysical observations of gravitational interactions provide strong evidence for the existence of dark matter. 
Many theories propose and experiments test the hypothesis that dark matter may have a particle physics origin, but this remains unproven. 
One such experiment is the Compact Muon Solenoid at the Large Hadron Collider. 
If dark matter couples, at least lightly, to the Standard Model, then it is possible to produce it in collisions at the LHC.
Because it would not interact with the detector, we must look for collisions in which dark matter is produced in association with one or more SM particles. 
This thesis describes two such analyses: dark matter plus one top quark and dark matter plus two light quarks.
Both cases result in complicated detector signatures due to the hadronization of final-state quarks. 
Recently developed jet substructure techniques were applied using novel methods to identify the hadronization products of high-momentum top quarks.
In both analyses, the observed data is found to be consistent with SM backgrounds.
We translate these results into the most stringent constraints to date on the relevant beyond-SM models.
