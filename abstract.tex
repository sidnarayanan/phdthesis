% $Log: abstract.tex,v $
% Revision 1.1  93/05/14  14:56:25  starflt
% Initial revision
% 
% Revision 1.1  90/05/04  10:41:01  lwvanels
% Initial revision
% 
%
%% The text of your abstract and nothing else (other than comments) goes here.
%% It will be single-spaced and the rest of the text that is supposed to go on
%% the abstract page will be generated by the abstractpage environment.  This
%% file should be \input (not \include 'd) from cover.tex.

Astrophysical observations of gravitational interactions provide strong evidence for the existence of dark matter (DM). 
Many theories propose and experiments test the hypothesis that DM may have a particle physics origin, but this remains unproven. 
One such experiment is the Compact Muon Solenoid (CMS) at the Large Hadron Collider (LHC). 
If DM couples to particles present in protons, it is possible that DM is produced in collisions at the LHC. 
Because DM, by its very nature, is effectively invisible to CMS, we must look for collisions in which DM is produced in association with one or more Standard Model (SM) particles. 
This thesis describes three different scenarios for the SM particle hypothesis: a single top quark, a single Higgs boson, or two light quarks. 
All three cases result in complicated detector signatures due to the hadronization of final-state quarks. 
Improved jet substructure techniques to identify these unique signatures are presented. 
Since the observed data is consistent with SM backgrounds in all three searches, we translate this result into the most stringent constraints to date on the relevant beyond-SM models.

