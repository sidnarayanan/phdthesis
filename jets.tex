\chapter{Hadronic Resonance Identification}
\label{sec:jets}

In this chapter, we describe the reconstruction and identification of heavy ($\gtrsim 100$ GeV) resonances that decay to two or more quarks.
Within the Standard Model, the only such resonances are the massive vector bosons ($W,Z\rightarrow q\bar{q}'$), the Higgs boson (typically $H\rightarrow b\bar{b}$), and the top quark ($t\rightarrow bW(\rightarrow q\bar{q}')$).
These quarks hadronize into jets (described in Chapter~\ref{sec:theory}), which are typically reconstructed at the LHC using the anti-$k_\mathrm{T}$ algorithm (described in Chapter~\ref{sec:cms}).
The focus of this chapter is on the cases in which the resonance is boosted and the decay products merge, such that they cannot be identified as 2 or 3 distinct jets.
In preparation for Chapter~\ref{sec:mt}, we will take the top quark as a concrete example.
In principle, the studies presented here can (and in some cases have been) applied to other heavy resonances, both within and beyond the Standard Model.

\section{Reconstruction}
\label{sec:jets:reco}

The approximate angular separation between the quarks from a heavy resonance decay is\needcite:
\begin{equation}
    \Delta R \sim \frac{2M}{\pt}
\end{equation}
where $M$ is the resonance mass and $\pt$ is the resonance transverse momentum.
Setting $M=m_t$ and $\Delta R=1.2$ (i.e. the radius at which three $R=0.4$ jets start to overlap), we extract a ``merging scale'' of $300$ GeV.  
This can be verified by checking the distribution of the ``decay radius'' in top quark simulation.
Here, we define decay radius as: 
\begin{equation}
    \max\Delta R_{qq} \equiv \displaystyle\max_{0\leq i < j \leq 2} \{\Delta R(q_i,q_j)\} \text{, where } t\rightarrow q_0q_1q_2
\end{equation}
Using a broad spectrum of generated top quark $\pt$, Figure~\ref{fig:jets:dr} shows the dependence of the decay radius on the top quark $\pt$, where we restrict the resonance to satisfy $|\eta|<2.5$.
If we are interested in top quarks with $\pt>250$ GeV (motivated by the trigger selection (Section~\ref{sec:mt:sel})), then over half of top quarks will be fully contained within a jet of radius $1.5$.
That is, at $\pt\approx 250$ GeV, it is equally likely that a top quark's decay products will fall within a single large-radius jet or that they will be resolvable as three separate jets. 
However, past this threshold momentum, the large-radius jet becomes the preferred reconstruction option.
This motivates the use of $R=1.5$ jets to reconstruct hadronic top quarks with $\pt>250$ GeV. 

\begin{figure}[]
    \begin{center}
        \includegraphics[width=0.75\textwidth]{figures/toptagging/gen/ptdr.pdf}
        \caption{Distribution of top quark momenta versus decay radii in a simulated top quark pair sample.
                 The events are weighted such that the inclusive momentum distribution is uniform. 
                 The $z$-axis units are arbitrary, but proportional to the distribution of jets. 
                 The solid red line marks the 50\% quantile of jets at each value of $\pt$. }
        \label{fig:jets:dr}
    \end{center}
\end{figure}

There are two tunable parameters in jet reconstruction.
We have specified the jet radius, but we must also choose the jet algorithm.
The anti-$k_\mathrm{T}$ algorithm tends to pick circular jets, whereas the Cambridge-Aachen (CA) algorithm allows for more geometric shapes (Figure~\ref{fig:jets:algos}).
As the top jets we seek to reconstruct are the sum of three light quark jets, we do not necessarily expect the $R=1.5$ jet to be circular.
Figure~\ref{fig:jets:caak} compares the jet mass distribution for top and light quark/gluon (QCD) jets, where the jets are clustered using both algorithms.
CA produces a top jet mass distribution with a narrower peak that sits closer to $m_t$ than anti-\kt. 
Because of this, and the general improvement in $S/B$ near the top mass peak, we choose the CA algorithm.
Hereafter, we will refer to Cambridge-Aachen $R=1.5$ jets as CA15 jets. 

\begin{figure}[]
    \begin{center}
        \includegraphics[width=0.35\textwidth]{figures/toptagging/gen/ak.png}
        \includegraphics[width=0.35\textwidth]{figures/toptagging/gen/ca.png}
        \caption{Jets clustered using the anti-$k_\mathrm{T}$ (left) and CA (right) algorithms.
                 Shown is the $y$-$\phi$ plane of a hypothetical calorimeter, unrolled onto a flat surface.
                 The height of each cell represents the \pt~of the particle. 
                 The anti-\kt~jets tend to be more circular when compared to the CA jets.
                 Figures are adapted from~\needcite.}
        \label{fig:jets:algos}
    \end{center}
\end{figure}

\begin{figure}[]
    \begin{center}
        \includegraphics[width=0.5\textwidth]{figures/toptagging/gen/clf_M.pdf}
        \caption{Mass distribution for jets clustered using the anti-$k_\mathrm{T}$ (dashed) and CA (solid) algorithms.
                 QCD refers to jets originating in QCD multijet events, i.e. from the hadronization of light quarks or gluons.}
        \label{fig:jets:caak}
    \end{center}
\end{figure}



\section{Identification}

Having \emph{reconstructed} the candidate top quark jets, we turn to the problem of \emph{identifying} which CA15 jets originate from top quarks as opposed to light $q/g$ hadronization. 
As indicated in Figure~\ref{figs:jets:caak}, the jet mass is a powerful observable, but top (QCD) jets do not necessarily have a mass of $m_t$ ($m_q,m_g\sim 0$). 
While some of this discrepancy is caused by mismeasurement of the jet energy scale (Chapter~\ref{sec:cms}), a substantial fraction originates from extra radiation being absorbed into the jet.
These extra particles can arise from pile-up (although this is accounted for by PUPPI), initial state radiation, and underlying event.
Many algorithms exist to ``groom'' such particles from a jet after it has been clustered; here, we will discuss the soft drop (SD) method.
SD functions by traversing the CA clustering history in reverse and removing branches of the clustering tree that are deemed to be too soft

\subsection{Substructure}

\subsubsection{A combined tagger}
\label{sec:jets:combined}

\subsection{Heavy flavor identification}

\section{Data validation}
