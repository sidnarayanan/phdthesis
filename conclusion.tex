\chapter{Conclusion}

Two searches for the production dark matter in association with jets at the Large Hadron Collider are conducted, using 36 \fbinv~ of 13 TeV proton-proton collision data collected by the Compact Muon Solenoid in 2016.

Previous searches for dark matter plus a top quark~\cite{cms8mt,atlas8mt} exclude scalar resonances with $m_\phi<330$ GeV and flavor changing neutral currents with $m_V<660$ GeV.
These analyses used 20 \fbinv~ of 8 TeV data and reconstructed the top quark decay products as three AK4 jets or 1 AK4 jet, a charged lepton, and missing momentum.
With a similar set of model assumptions, the results presented here exclude models with $m_\phi < 3.5$ TeV and $200<m_V<1.8$ TeV.
This marked increase in sensitivity is driven by improvements in top jet reconstruction and invisible background estimation.
The higher collision energy and integrated luminosity also increase the expected number of signal events. 
Future searches for this signature will benefit greatly from continued work in top jet identification, as the primary backgrounds are still $V$+jet processes.

Prior direct seaches for invisible Higgs decays~\cite{cmshinvrun1} are able to reach observed (expected) upper limits of $\mathcal{B}(\hinv) < 0.24~(0.23)$, using a 26.9 \fbinv~ dataset from Run 1 and Run 2 of the LHC.
This result, using an orthogonal dataset, places a constraint of $\mathcal{B}(\hinv) < 0.26~(0.20)$.
Despite the upwards fluctuation, the improvement in the expected limit is driven by the improved estimation of EW and QCD $V$+jets processes.
The L1 trigger effects, which result in up to 20\% of the signal being lost, have been fixed in the remainder of the Run 2 dataset.
Future searches at CMS will therefore benefit from both the analysis techniques presented in this result and the improved trigger.

Both the mono-top and VBF \hinv~searches represent the strongest expected constraints to date on the relevant DM hypotheses.
The analyses are designed to be as model-independent as possible, instead focusing on general characteristics of the interesting final states. 
We do so because the precise nature of DM is not well-understood, and the greatest impact can be had by simultaneously probing an entire class of models, as opposed to very specific signatures.
The results of this thesis can be re-interpreted in the context of any DM model that couples the first and third generation of quarks or any DM model that generates mass through a Higgs mechanism. 

The precise estimation of SM processes containing a single vector boson and multiple jets is critical for both analyses.
Despite the data-driven background prediction strategy, we still rely on theoretical calculations of the ratios of various $V$+jets spectra.
Dedicated calculations targeting the jet multiplicity and phase space probed by these searches will reduce the uncertainties on the ratio prediction.
This will have a profound impact on the sensitivity to signatures searched for in these results, as well as on other searches for jets plus missing momentum~\cite{monojet,monohiggs,ttdm}.

The remainder of the Run 2 dataset provides an ample opportunity to continue to improve analysis techniques and better probe dark matter hypotheses.
Beyond Run 2, Run 3 and the High Luminosity LHC will offer improved detectors and larger integrated luminosities, which can further augment our sensitivity to beyond-SM physics.
