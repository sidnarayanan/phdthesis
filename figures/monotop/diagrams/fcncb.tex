% template.tex -- trivial FeynMF example
% $Id: template.tex,v 1.1 1996/12/02 01:38:45 ohl Exp $
% To typeset, either use the feynmf script
%   $ feynmf template
% or run Metafont manually
%   $ latex template
%   $ mf '\mode:=localfont; input fmftempl'
%   $ latex template

\documentclass{beamer}
\usepackage{feynmp}%%%{feynmp}
\usepackage{amsmath}
\usefonttheme[onlymath]{serif}
\unitlength=1mm
\setbeamertemplate{navigation symbols}{}
\definecolor{histbackground}{RGB}{255,255,255}
\setbeamercolor{background canvas}{bg=histbackground}
\setlength{\paperwidth}{9cm}
\setlength{\paperheight}{5.4cm}
\setlength{\textwidth}{9cm}
\setlength{\textheight}{5.4cm}
\begin{document}
\Large
\begin{frame}{ }
\vspace{2mm}
\begin{fmffile}{mjb}
  \begin{fmfchar*}(70,40)
  \fmfstraight
    \fmfleft{i2,i1}  \fmflabel{$u$}{i1} \fmflabel{$g$}{i2}
    \fmfright{o1,o2,o3}  \fmflabel{$\chi$}{o3} \fmflabel{$\bar\chi$}{o2} \fmflabel{$t$}{o1}
  \fmf{curly,tension=1.5}{i2,v1}
    \fmf{fermion,label=$u$,tension=2}{v1,v2}
    \fmf{photon,tension=2,label=$V$,label.side=left}{v2,v3}
  % \fmffreeze
  \fmf{fermion,tension=1.5}{i1,v1}
  \fmf{fermion}{v2,o1}
  \fmf{fermion,tension=3}{v3,o3}
  \fmf{fermion}{o2,v3}
  % \fmffreeze
\end{fmfchar*}
\end{fmffile}

\end{frame}
\end{document}

